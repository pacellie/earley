%
\begin{isabellebody}%
\setisabellecontext{{\isadigit{0}}{\isadigit{7}}{\isacharunderscore}{\kern0pt}Conclusion}%
%
\isadelimtheory
%
\endisadelimtheory
%
\isatagtheory
%
\endisatagtheory
{\isafoldtheory}%
%
\isadelimtheory
%
\endisadelimtheory
%
\isadelimdocument
%
\endisadelimdocument
%
\isatagdocument
%
\isamarkupchapter{Conclusion%
}
\isamarkuptrue%
%
\isamarkupsection{Summary%
}
\isamarkuptrue%
%
\endisatagdocument
{\isafolddocument}%
%
\isadelimdocument
%
\endisadelimdocument
%
\begin{isamarkuptext}%
TODO%
\end{isamarkuptext}\isamarkuptrue%
%
\isadelimdocument
%
\endisadelimdocument
%
\isatagdocument
%
\isamarkupsection{Future Work%
}
\isamarkuptrue%
%
\endisatagdocument
{\isafolddocument}%
%
\isadelimdocument
%
\endisadelimdocument
%
\begin{isamarkuptext}%
TODO: collect implementation improvements, highlight worthwhile alternative implementation ideas,
  proof improvements, and further algorithms such as SPPFs%
\end{isamarkuptext}\isamarkuptrue%
%
\begin{isamarkuptext}%
Parse tree disambiguation:

Parser generators like YACC resolve ambiguities in context-free grammers by allowing the user
the specify precedence and associativity declarations restricting the set of allowed parses. But they
do not handle all grammatical restrictions, like 'dangling else' or interactions between binary operators
and functional 'if'-expressions.

Grammar rewriting:

Adams \textit{et al} \cite{Adams:2017} describe a grammar rewriting approach reinterpreting CFGs as
the tree automata, intersectiong them with tree automata encoding desired restrictions and reinterpreting
the results back into CFGs.

Afroozeh \textit{et al} \cite{Afroozeh:2013} present an approach to specifying operator precedence
based on declarative disambiguation rules basing their implementation on grammar rewriting.

Thorup \cite{Thorup:1996} develops two concrete algorithms for disambiguation of grammars based on the idea of 
excluding a certain set of forbidden sub-parse trees.

Parse tree filtering:

Klint \textit{et al} \cite{Klint:1997} propose a framework of filters to describe and compare a wide
range of disambiguation problems in a parser-independent way. A filter is a function that selects
from a set of parse trees the intended trees.%
\end{isamarkuptext}\isamarkuptrue%
%
\isadelimtheory
%
\endisadelimtheory
%
\isatagtheory
%
\endisatagtheory
{\isafoldtheory}%
%
\isadelimtheory
%
\endisadelimtheory
%
\end{isabellebody}%
\endinput
%:%file=07_Conclusion.tex%:%
%:%24=8%:%
%:%28=10%:%
%:%40=13%:%
%:%49=16%:%
%:%61=19%:%
%:%62=20%:%
%:%66=24%:%
%:%67=25%:%
%:%68=26%:%
%:%69=27%:%
%:%70=28%:%
%:%71=29%:%
%:%72=30%:%
%:%73=31%:%
%:%74=32%:%
%:%75=33%:%
%:%76=34%:%
%:%77=35%:%
%:%78=36%:%
%:%79=37%:%
%:%80=38%:%
%:%81=39%:%
%:%82=40%:%
%:%83=41%:%
%:%84=42%:%
%:%85=43%:%
%:%86=44%:%
%:%87=45%:%
%:%88=46%:%
%:%89=47%:%
